% Define document class
\documentclass[twocolumn]{aastex63}
\DeclareRobustCommand{\Eqref}[1]{Eq.~\ref{#1}}
\DeclareRobustCommand{\Figref}[1]{Fig.~\ref{#1}}
\DeclareRobustCommand{\Tabref}[1]{Tab.~\ref{#1}}
\DeclareRobustCommand{\Secref}[1]{Sec.~\ref{#1}}
\newcommand{\todo}[1]{{\large $\blacksquare$~\textbf{\color{red}[#1]}}~$\blacksquare$}
% \usepackage{cuted}
\usepackage{flushend}
\usepackage{amsmath}
\usepackage{showyourwork}
\graphicspath{{./figures/}}

\begin{document}

% Title
\title{Pair-instability mass loss for top-down compact object mass calculations}

\author[0000-0002-6718-9472]{M.~Renzo}
\affiliation{Center for Computational Astrophysics, Flatiron Institute, New York, NY 10010, USA}
\affiliation{Department of Physics, Columbia University, New York, NY 10027, USA}

\author[0000-0002-8717-6046]{D.~D.~Hendriks}
\affiliation{Department of Physics, University of Surrey, Guildford, GU2 7XH, Surrey, UK}

\author[0000-0001-5484-4987]{L.~A.~C.~van~Son}
\affiliation{Center for Astrophysics $|$ Harvard $\&$ Smithsonian,60 Garden St., Cambridge, MA 02138, USA}
\affiliation{Anton Pannekoek Institute for Astronomy, University of Amsterdam, Science Park 904, 1098XH Amsterdam, The Netherlands}
\affiliation{Max-Planck-Institut für Astrophysik, Karl-Schwarzschild-Straße 1, 85741 Garching, Germany}

\author[0000-0003-3441-7624]{R.~Farmer}
\affiliation{Max-Planck-Institut für Astrophysik, Karl-Schwarzschild-Straße 1, 85741 Garching, Germany}

\begin{abstract}
  \noindent
  Population synthesis relies on semi-analytic formulae to determine
  masses of compact objects from the (helium or carbon-oxygen) cores
  of collapsing stars. Such formulae are combined across mass ranges
  that span different explosion mechanisms, potentialy introducing
  artificial features in the compact object mass distribution. Such
  artifacts impair the interpretation of gravitational-wave
  observations. We propose a ``top-down'' remnant mass prescription
  where we remove mass from the star for each possible mass-loss
  mechanism, instead of relying on the fallback onto a
  ``proto-compact-object'' to get the final mass. For one of these
  mass-loss mechanisms, we fit the metallicity-dependent mass lost to
  pulsational-pair instability supernovae from numerical
  simulations. By imposing no mass loss in the absence of pulses, our
  approach recovers the existing compact object masses at the low mass
  end and ensures continuity across the
  core-collapse/pulsational-pair-instability regime. % Our remnant mass
  % prescription can be extended
  % to include other mass-loss mechanisms at the final collapse.\\
\end{abstract}

\section{Introduction}

Stellar and binary population synthesis calculations are necessary to
predict event rates and population statistics of astrophysical
phenomena, including those involving neutron stars (NS) and
black holes (BH). Typically, at the end of the evolution (carbon
depletion) the mass of the core is mapped to a compact object mass, using a
$M_\mathrm{comp.\ obj}\equiv M_\mathrm{comp.\ obj}(M_\mathrm{core})$
informed by core-collapse (CC) simulations (e.g., \citealt{fryer:12,
  spera:15, mandel:20, couch:20}, see also \citealt{zapartas:21,
  patton:21}) and/or (pulsational) pair instability (PPI) simulations
(e.g., \citealt{belczynski:16, woosley:17, spera:17, stevenson:19,
  marchant:19, farmer:19, breivik:20, renzo:20csm, costa:21}).

The most commonly adopted algorithms to obtain compact object masses
in the CC regime are the ``rapid'' and ``delayed'' prescriptions of
\cite{fryer:12}. In both cases, the compact object is built from the
``bottom-up'', starting from a proto-NS mass and adding the amount of
fallback expected in the (possibly failed) explosion. However, the
proto-NS mass and information about the core structure relevant to
calculate the fallback are usually not available
\citep[e.g.,][]{patton:20}. Instead, the total final mass of the star
is arguably easier to constrain in population synthesis models.

At the transition between CC and PPI (roughly at carbon-oxygen cores
of $\sim{}35\,M_\odot$, \citealt{woosley:17, marchant:19, farmer:19,
  renzo:20csm, costa:21}), a mismatch between commonly adopted fitting
formulae exists, and impairs the
interpretation of gravitational-wave data \citep[as pointed out in
Fig.~5 of][]{vanson:21}. While it is possible that the BH mass
function is discontinuous at the onset of the PPI regime (e.g.,
\citealt{renzo:20conv,costa:21}, Hendriks et al., in prep.), the
location and amplitude of a putative discontinuity should not be
governed by a mismatch between the fitting formulae.


\section{Top-down compact object masses}

In contrast with the ``bottom up'' approach of \cite{fryer:12}, we
propose a ``top-down'' compact object mass calculation. Starting from
the total stellar mass, we remove the amount of mass lost due to all
of the processes associated with the (possibly failed) explosion:

\begin{widetext}
  \begin{equation}
    \label{eq:mass}
      M_\mathrm{comp.\ obj} =
      M_\mathrm{pre-CC} - \left(\Delta M_\mathrm{SN} + \Delta M_{\nu, \mathrm{core}} + \Delta M_\mathrm{env} + \Delta M_\mathrm{PPI} + \cdots \right)
  \end{equation}
\end{widetext}

% where all masses are in $M_\odot$,
where $M_\mathrm{pre-CC}$ is the total mass at the onset of CC, and
each term in parenthesis corresponds to a potential mass-loss
mechanism: $\Delta M_\mathrm{SN}$ for the CC ejecta,
$\Delta M_{\nu, \mathrm{core}}$ the change in gravitational mass of
the core due to the neutrino losses, $\Delta M_\mathrm{env}$ the loss
of the envelope that can occur even in red supergiant ``failed''
core-collapse due to % the change in
% gravitational mass corresponding to
$\Delta M_{\nu, \mathrm{core}}$
\citep{nadezhin:80, lovegrove:13, piro:13, fernandez:18, ivanov:21},
and $\Delta M_\mathrm{PPI}$ the pulsational mass loss due to
pair-instability. Each term may be a function of the progenitor
properties, and may be theoretically or observationally informed
(e.g., $\Delta M_\mathrm{SN}$ could be derived from the light curves of a
large sample of observed SNe). \Eqref{eq:mass} can be extended by
adding additional mass-loss mechanisms in the parenthesis (e.g., disk
winds).

In the CC regime, the previous approach from \cite{fryer:12} can be recovered by setting
$\Delta M_\mathrm{SN} + \Delta M_{\nu, \mathrm{core}} = M_\mathrm{pre-CC} - M_\mathrm{comp.\ obj}^\mathrm{Fryer+12}$,
where the last term is the compact object mass as predicted by
\cite{fryer:12} and ignoring the other mass loss terms,
such as $\Delta M_\mathrm{PPI}$ and $\Delta M_\mathrm{env}$.

\begin{figure}[tbp]
    \begin{centering}
      \includegraphics[width=0.75\linewidth]{fit_DM_PPI.pdf}
      \caption{Each panel shows our fitting formula \Eqref{eq:fit} for
        the PPI induced mass-loss as a function of carbon-oxygen core
        mass at each metallicity $Z$. The crosses show the values from
        Tab.~1 \cite{farmer:19}.}
        \label{fig:fit_DM_PPI}
        % This command tells showyourwork that the script `figures/fit_DM_PPI.py'
        % generates the PDF included above
        \script{fit_DM_PPI.py}
    \end{centering}
\end{figure}

\section{New fit for PPI ejecta}


Imposing $\Delta M_\mathrm{PPI}=0$ at the edge of the PPI regime,
\Eqref{eq:mass} produces a smooth BH mass distribution.
\Eqref{eq:fit} provides a fit (in $M_\odot$ units) to naked helium
star models from \cite{farmer:19} for
$\Delta M_\mathrm{PPI} \equiv \Delta M_\mathrm{PPI}(M_\mathrm{CO},Z)$.
While the fit of \cite{farmer:19} provides the BH mass after PPI, this
is only an estimate because of other mass loss processes that might
occur at CC \citep[e.g.,][]{renzo:20csm, powell:21, rahman:22}. Here,
we fit the mass removed by PPI (crosses in \Figref{fig:fit_DM_PPI}),
which is what is directly computed in \cite{farmer:19}.

% Each panel of \Figref{fig:fit_DM_PPI} shows
% % The dashed curves in each panel of \Figref{fig:fit_DM_PPI} show the
% % fit \Eqref{eq:fit} for each
% one metallicity computed in
% \cite{farmer:19}.
We neglect the (weak) metallicity dependence of the minimum core mass
for PPI, and we fit the data for initial helium core masses between
$38-60\,M_\odot$.  We emphasize that \cite{farmer:19} only simulated
helium cores. In the presence of a H-rich envelope at the onset of
PPI, if it is extended and red it can be easily removed by the first
pulse \citep[][]{woosley:17,renzo:20csm}. Thus the H-rich mass of red
supergiants should be added to the $\Delta M_\mathrm{PPI}$ provided
here. It is unclear what occurs in cases when the envelope is compact
and blue \citep[e.g.,][]{dicarlo:19, renzo:20merger, costa:21}.

\variable{output/fit_DM_PPI.tex} %% generated by src/scripts/fit_DM_PPI.py

The mass lost in PPI is sensitive to convection
\citep{renzo:20conv} and nuclear physics \citep{farmer:19,
  farmer:20, costa:21, woosley:21, mehta:21}, which can introduce
uncertainties up to $\sim{}20\%$ on the maximum BH mass. The accuracy
of our fit is comparable to these uncertainties.

\vspace*{-10pt}

% \section*{Acknowledgements}
% MR is grateful to R.~Luger for help with showyourwork
% \citep{luger:21}. The code associated to this paper is publicly
% available at
% \url{https://github.com/mathren/top_down_compact_obj_mass} and the
% input data are loaded from \url{https://zenodo.org/record/3346593}.  LvS
% acknowledges partial financial support from the National Science
% Foundation under Grant No. (NSF grant number 2009131), the Netherlands
% Organisation for Scientific Research (NWO) as part of the Vidi
% research program BinWaves with project number 639.042.728 and the
% European Union’s Horizon 2020 research and innovation program from the
% European Research Council (ERC, Grant agreement No. 715063).


\newpage
\bibliography{./top_down_comp_obj_mass.bib}
\end{document}

%%% Local Variables:
%%% mode: latex
%%% TeX-master: t
%%% End:
