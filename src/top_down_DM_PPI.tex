% Define document class
\documentclass[twocolumn]{aastex63}
\DeclareRobustCommand{\Eqref}[1]{Eq.~\ref{#1}}
\DeclareRobustCommand{\Figref}[1]{Fig.~\ref{#1}}
\DeclareRobustCommand{\Tabref}[1]{Tab.~\ref{#1}}
\DeclareRobustCommand{\Secref}[1]{Sec.~\ref{#1}}
\newcommand{\todo}[1]{{\large $\blacksquare$~\textbf{\color{red}[#1]}}~$\blacksquare$}
% \usepackage{cuted}
\usepackage{flushend}

\graphicspath{{./figures/}}

\begin{document}

% Title
\title{Pair-instability mass loss for top-down compact object mass calculations}

% Author list
\author[0000-0002-6718-9472]{M.~Renzo}
\affiliation{Center for Computational Astrophysics, Flatiron Institute, New York, NY 10010, USA}
\affiliation{Department of Physics, Columbia University, New York, NY 10027, USA}


\author[0000-0001-5484-4987]{L.~A.~C.~van~Son}
\affiliation{Center for Astrophysics $|$ Harvard $\&$ Smithsonian,60 Garden St., Cambridge, MA 02138, USA}
\affiliation{Anton Pannekoek Institute for Astronomy, University of Amsterdam, Science Park 904, 1098XH Amsterdam, The Netherlands}
\affiliation{Max Planck Institute for Astrophysics, Karl-Schwarzschild-Str. 1, 85748 Garching, Germany}


\author[0000-0002-8717-6046]{D.~D.~Hendriks}
\affiliation{Department of Physics, University of Surrey, Guildford, GU2 7XH, Surrey, UK}

\author{order TBD}











\begin{abstract}
  \noindent
  Population synthesis relies on semi-analytic formulae to determine
  masses of compact objects from the (helium or carbon-oxygen) core of
  collapsing stars. Typically, such formulae are combined across mass
  ranges that span different explosion mechanisms, possibly
  introducing artificial features in the compact object mass
  distribution. Such artifacts impair the interpretation of
  gravitational-wave observations. Instead of relying on the fallback
  onto a ``proto-compact-object'' to get the final mass, we propose a
  ``top-down'' remnant mass prescription where we remove mass from the
  star for each possible mass-loss mechanism. For one of these, we fit
  the metallicity-dependent mass lost to pulsational-pair instability
  in numerical simulations. By imposing no mass loss in the absence of
  pulses, our approach recovers the existing compact object mass
  prescription at the low mass end, ensures continuity across the
  core-collapse/pulsational-pair-instability regime. Our remnant mass
  prescription can be extended
  to other mass-loss mechanisms associated with the final collapse.\\
\end{abstract}

\section{Introduction}

Stellar and binary population synthesis calculations are necessary to
predict event rates and population statistics of astrophysical
phenomena involving neutron stars (NS) and
black holes (BH). Typically, at the end of the evolution (carbon
depletion) the core mass is mapped to a compact object mass, using a
$M_\mathrm{comp.\ obj}\equiv M_\mathrm{comp.\ obj}(M_\mathrm{core})$
informed by core-collapse (CC) simulations (e.g., \citealt{fryer:12,
  spera:15, mandel:20, couch:20}, see also \citealt{zapartas:21,
  patton:21}) and/or (pulsational) pair instability (PPI) simulations
(e.g., \citealt{belczynski:16, woosley:17, spera:17, stevenson:19,
  marchant:19, farmer:19, costa:21}).

The most commonly adopted algorithms to obtain compact object masses
in the CC regime are the ``rapid'' and ``delayed''
prescriptions from \cite{fryer:12}. In both cases, the compact object
is built ``bottom-up'', starting from a proto-NS mass and adding the amount of
fallback expected in the (possibly failed) explosion. However, the
proto-NS mass, and information about the core structure relevant
to decide the fallback are usually not available in population
synthesis \citep[e.g.,][]{patton:20} -- conversely the total final mass of
the star is arguably easier to constrain. % Various algorithms are used for the
% PPI regime, including \cite{belczynski:16, stevenson:19, marchant:19},
% and \cite{farmer:19}.

At the transition between CC and PPI (roughly at carbon-oxygen cores
of $\sim{}35\,M_\odot$, \citealt{woosley:17, marchant:19, renzo:20csm,
  costa:21}), a mismatch between commonly adopted fitting formulae
exists. This are at least partly caused by differences in the stellar structures used to
design these algorithms and impair the
interpretation of gravitational-wave data \citep[as pointed out in
Fig.~5 of][]{vanson:21}. While it is possible that the BH mass
function is discontinuous at the onset of the PPI regime (e.g.,
\citealt{renzo:20conv,costa:21}, Hendriks et al., in prep.), the
location and amplitude of a putative discontinuity should not be
governed by a mismatch between fitting formulae to different stellar
evolution models.

\section{Top-down compact object masses}

In contrast with the  ``bottom up'' approach of
\cite{fryer:12}, we propose a ``top-down'' way to build the compact object masses
$M_\mathrm{comp.\ obj}$, starting from the total stellar mass and
removing the amount of mass lost because of all the processes
associated with the (possibly failed) explosion:

% \todo{The most common algorithm are from Fryer+12, where the compact
%   object mass if build from a proto-NS mass adding fallback (bottom-up).}\\
% \todo{pop. synth. knows total masses, while core masses are poorly
%   characterized (e.g., lacking composition informaton,
%   \citealt{patton:20}). While counterintuitive to core-collapse
%   theorists, it is easier to build the compact object core mass from
%   the total mass at pre-collapse minus all the SN mass loss term
%   (top-down):}\\

\begin{widetext}
  \begin{equation}
    \label{eq:mass}
    M_\mathrm{comp.\ obj} = M_\mathrm{pre-CC} - \left(\Delta M_\mathrm{SN} + \Delta M_{\nu, \mathrm{core}} + \Delta M_\mathrm{env} + \Delta M_\mathrm{PPI} + \cdots \right)
  \end{equation}
\end{widetext}

\begin{figure}[ht!]
    \begin{centering}
      \includegraphics[width=0.75\linewidth]{fit_DM_PPI.pdf}
      \caption{Each panel show our fitting formula \Eqref{eq:fit} for
        the amount of PPI mass-loss as a function of carbon-oxygen
        core mass at each metallicity $Z$ computed in
        \cite{farmer:19}. The crosses show the values from Tab.~1
        \cite{farmer:19}.}
        % This label tells showyourwork that the script `figures/fit_DM_PPI.py'
        % generates the PDF included above
        \label{fig:fit_DM_PPI}
    \end{centering}
\end{figure}

where all masses are in $M_\odot$ units, $M_\mathrm{pre-CC}$ is the
total mass at the onset of CC, and each term in the parenthesis
correspond to a potential mass-loss mechanism: $\Delta M_\mathrm{SN}$
for the CC ejecta, $\Delta M_{\nu, \mathrm{core}}$ the change in
gravitational mass of the core due to the neutrino losses,
$\Delta M_\mathrm{env}$ the loss of the envelope due to the change in
gravitational mass corresponding to $\Delta M_{\nu, \mathrm{core}}$
that can occur even in red supergiant ``failed'' core-collapse
\citep{nadezhin:80, lovegrove:13, piro:13, fernandez:18, ivanov:21},
and $\Delta M_\mathrm{PPI}$ the pulsational mass loss due to
pair-instability. Each term can be a function of the stellar or binary
properties, and can be theoretically or observationally informed
(e.g., $\Delta M_\mathrm{SN}$ might be derived from light curves of
large samples of observed SNe). \Eqref{eq:mass} can be extended
adding different mass-loss mechanisms in the parenthesis (e.g., disk
winds).

In the CC regime, the previous approach can be recovered by setting
$\Delta M_\mathrm{SN} + \Delta M_{\nu, \mathrm{core}} = M_\mathrm{pre-CC} - M_\mathrm{comp.\ obj}^\mathrm{Fryer+12}$,
where the last term is the compact object mass predicted by
\cite{fryer:12} and ignoring other mass loss terms
such as $\Delta M_\mathrm{PPI}$ and $\Delta M_\mathrm{env}$.

\section{New fit for PPI ejecta}

The top-down approach of \Eqref{eq:mass}, imposing that
$\Delta M_\mathrm{PPI}=0$ at the edge of the PPI regime
produces by construction a smooth BH mass
distribution. We fit the metallicity-dependent naked helium core PPI
simulations of \cite{farmer:19} to obtain
$\Delta M_\mathrm{PPI} \equiv \Delta M_\mathrm{PPI}(M_\mathrm{CO},Z)$ as
a function of the carbon-oxygen core mass (\Figref{fig:fit_DM_PPI}). While the fit of
\cite{farmer:19} provides the remaining mass after PPI, this is only
an estimate of the BH mass because of other mass loss processes that
might occurr \citep[e.g.,][]{renzo:20csm, powell:21, rahman:22}. Our approach
fits the mass removed by PPI only, which is what is directly computed
in \cite{farmer:19}.

\begin{widetext}
\begin{equation}
\label{eq:fit}
\Delta M_\mathrm{PPI} = (0.0006\log_{10}(Z)+0.0054)\times (M_\mathrm{CO}-34.8)^3-0.0013\times (M_\mathrm{CO}-34.8)^2
\end{equation}
\end{widetext}
 %% generated by src/figures/fit_DM_PPI.py

The dashed curves in each panel of \Figref{fig:fit_DM_PPI} show the
fit \Eqref{eq:fit} for each metallicity computed in
\cite{farmer:19}. We neglect the (weak) metallicity dependence of the
minimum core mass for PPI, and we fit the publicly available data for initial He core
masses between $38-60\,M_\odot$.  We emphasize that \cite{farmer:19}
only simulated helium cores. In case an H-rich envelope is still
present at the onset of pulsation, if it is extended and red, it
is easily removed by the first pulse \citep[][]{woosley:17,renzo:20csm} and the H-rich mass should be added to the
$\Delta M_\mathrm{PPI}$ provided here. It is unclear what occurs in
cases when the envelope is compact and blue \cite{dicarlo:19, renzo:20merger, costa:21}.

The amount of mass lost in PPI is sensitive to convection
\cite{renzo:20conv} and nuclear reaction rates \cite{farmer:19,
  farmer:20, costa:21, woosley:21, mehta:21}, which overall introduce
up to $\sim{}20\%$ uncertainties on the maximum BH mass. The accuracy
of our fit is comparable to these uncertainties.

\section*{Acknowledgements}
LvS acknowledges partial financial support from the  National Science Foundation under Grant No. (NSF grant number 2009131), the Netherlands Organisation for Scientific Research (NWO) as part of the Vidi research program BinWaves with project number 639.042.728 and the European Union’s Horizon 2020 research and innovation program from the European Research Council (ERC, Grant agreement No. 715063).


\newpage
\bibliography{./top_down_comp_obj_mass.bib}
\end{document}

%%% Local Variables:
%%% mode: latex
%%% TeX-master: t
%%% End:
