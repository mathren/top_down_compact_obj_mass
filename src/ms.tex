% Define document class
\documentclass[twocolumn]{aastex63}
\DeclareRobustCommand{\Eqref}[1]{Eq.~\ref{#1}}
\DeclareRobustCommand{\Figref}[1]{Fig.~\ref{#1}}
\DeclareRobustCommand{\Tabref}[1]{Tab.~\ref{#1}}
\DeclareRobustCommand{\Secref}[1]{Sec.~\ref{#1}}
\newcommand{\todo}[1]{{\large $\blacksquare$~\textbf{\color{red}[#1]}}~$\blacksquare$}

\begin{document}

% Title
\title{Top-down compact object masses for population synthesis}

% Author list
% \author{M. Renzo}

\begin{abstract}
  Stellar (binary) population synthesis codes use semi-analytic
  formulae to determine masses of compact objects from the (helium or
  carbon-oxygen) core masses of collapsing stars. We propose a
  different implementation compatible with the existing approach which
  imposes by construction continuity across the
  core-collapse/pair-instability range -- to prevent artificial
  artifact -- and is easily extendable to other supernova mass loss
  mechanisms.
\end{abstract}

\section*{}
\todo{edit all at wish. Look also in the readme}
Stellar and binary population synthesis calculations are a necessary
tool to predict event rates and population statistics of astrophysical
phenomena involving compact objects (especially neutron stars and
black holes). Typically, at the end of the evolution (carbon
depletion) the core mass is mapped to a compact object mass, using an
% $M_\mathrm{comp. obj}\equiv M_\mathrm{comp.\ obj}(M_\mathrm{core})$
informed by core-collapse simulations (e.g., \citealt{fryer:12,
  spera:15, mandel:20, couch:20}, see also \citealt{zapartas:21, patton:21}) and/or (pulsational) pair instability simulations
(e.g., \citealt{belczynski:16, woosley:17, stevenson:19, marchant:19, farmer:19}). \todo{cite
  mapelli's groups}

For carbon-oxygen core masses of $\sim{}35\,M_\odot$, corresponding
roughly to the transition between core-collapse and pulsational
pair-instability, a mismatch between the fitting formulate can
introduce artificial features which presently impair the
interpretation of gravitational wave data \citep[see e.g. the
corresponding feature in Fig.~5 of][]{vanson:21}.

\todo{discontinuity can be real -- cite Renzo 2020 and Mapelli's
  merger. We plan to study it in a separate paper (Hendriks et al., in
  prep).}\\

\todo{The most common algorithm are from Fryer+12, where the compact
  object mass if build from a proto-NS mass adding fallback (bottom-up).}\\
\todo{pop. synth. knows total masses, while core masses are poorly
  characterized (e.g., lacking composition informaton,
  \citealt{patton:20}). While counterintuitive to core-collapse
  theorists, it is easier to build the compact object core mass from
  the total mass at pre-collapse minus all the SN mass loss term
  (top-down):}\\

\begin{equation}
  \label{eq:mass}
  M_\mathrm{comp.\ obj} = M_\mathrm{pre-CC} - \left(\Delta M_\mathrm{SN} + \Delta M_\mathrm{NLW} + \Delta M_\mathrm{PPI} + \Delta M_{\nu, \mathrm{core}} +\Delta M_\mathrm{lGRB} + \cdots \right)
\end{equation}

\todo{each term can be a function of stellar parameters known to the
  pop. synth.}\\

\todo{No refitting $\Delta M_\mathrm{SN}$ needed, and can be
  generalized to empirically informed SN-ejecta distribution mass --
  give ref}\\

\todo{list other mechanism with refs}\\

\todo{We present here a fitting formula that imposes continuity by
  substituting the fit to Farmer+19 -- instead of fitting the BH mass
  we fit what MESA actually calculates, which is the amount of mass
  lost.}

% A sample figure generated from an external dataset
\begin{figure}[ht!]
    \begin{centering}
      \includegraphics[width=0.75\linewidth]{figures/fit_DM_PPI.pdf}
        \caption{figure }
        % This label tells showyourwork that the script `figures/fit_DM_PPI.py'
        % generates the PDF included above
        \label{fig:fit_DM_PPI}
    \end{centering}
\end{figure}

\todo{this formula neglects the (weak) Z-dependence of the minimum CO
  core mass for PPISN (Farmer+19), and cannot be directly physically
  interpreted because of the artificially imposed conditions.}

\todo{fix bibtex arxiv entries}
% \bibliographystyle{aasjournal}
\bibliography{./bib.bib}



\end{document}

%%% Local Variables:
%%% mode: latex
%%% TeX-master: t
%%% End:
